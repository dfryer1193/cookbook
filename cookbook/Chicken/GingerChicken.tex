\begin{recipe}
[ %
    preparationtime = {\unit[???]{?}},
    portion = {\portion{4 small plate}},
    calory={???},
]
{Ginger Chicken}

    \graph
    {% pictures
        %small=pic/glass,     % small picture
        %big=pic/ingredients  % big picture
    }
    \ingredients
    {%
        \unit[4]{tb}     & Dry Sherry \\
        \unit[3]{tb}     & Soy Sauce \\
        \unit[1]{tb}     & Water \\
        \unit[1]{tsp}    & cornstarch\\
        \unit[1]{lb}     & Chicken Breast (bite sized pieces)\\
        \unit[2]{tb}     & Cooking Oil  \\
        \unit[1]{}       & Medium Ginger Root \\
        \unit[1.5]{cups} & Bias-sliced Carrots\\
        \unit[3.5]{cups} & Sliced Bok Choy \\
        \unit[2]{cups}   & Fresh pea pods\\
        \unit[1]{bunch}  & Scallions \\
        \unit[1]{}       & Habanero \\
        \unit[1.5]{cups} & Cooked Rice
    }
    \preparation
    {%
        \step Stir together sherry, soy sauce, water, and cornstarch.
        \step Heat a wok or large skillet over high heat, with 1 tb of oil
        \step Stir-fry ginger root for about 45 seconds
        \step Add carrots, and stir-fry until tender (about 5 minutes)
        \step Add bok choy, pea pods, and scallions, then stir fry until bok choy wilts (about 5-7 minutes)
        \step Remove vegetables from the wok, and place in separate bowl
        \step Add remaining cooking oil, and cook chicken in wok until done (about 5 minutes)
        \step Push chicken from center of wok, stir soy sauce mixture, and add to center of wok
        \step Cook until mixture is thickened and bubbly
        \step Return vegetables to wok, and stir to coat with sauce
        \step Cook and stir about 1 minute or until heated through, then serve
    }

    \hint
    {%
        ??? calories per thingie
    }
\end{recipe}
